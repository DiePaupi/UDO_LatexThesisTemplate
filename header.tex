% header.tex
\documentclass[a4paper,11pt,english,oneside]{book}
\usepackage[a4paper,left=2.5cm,right=2.5cm,bottom=2.5cm,top=4cm, headsep=1.5cm]{geometry}

%%%%%%%%%%%%%%%%%%%%%%%%%%%%%%%%%%%%%%%%%%%%%%%%%%%%%%%
% Basic packages
%%%%%%%%%%%%%%%%%%%%%%%%%%%%%%%%%%%%%%%%%%%%%%%%%%%%%%%
\usepackage[english]{babel}
\usepackage[pdftex]{graphicx}
\usepackage[dvipsnames]{xcolor}
\usepackage{amsmath,amssymb,subfigure}
\usepackage{enumerate}
\usepackage{lipsum} 

\usepackage[utf8]{inputenc}
\newcommand*{\Scale}[2][4]{\scalebox{#1}{$#2$}} % Allows equations to be scaled

\usepackage{csquotes} % Addition to babel package; Needs to be run after loading the package "inputenc"


%%%%%%%%%%%%%%%%%%%%%%%%%%%%%%%%%%%%%%%%%%%%%%%%%%%%%%%
% Font
%%%%%%%%%%%%%%%%%%%%%%%%%%%%%%%%%%%%%%%%%%%%%%%%%%%%%%%
\usepackage[lf]{Baskervaldx}				% lining figures
\usepackage[bigdelims,vvarbb]{newtxmath}	% math italic letters from Nimbus Roman
\usepackage[cal=boondoxo]{mathalfa}			% mathcal from STIX, unslanted a bit
\renewcommand*\oldstylenums[1]{\textosf{#1}}


%%%%%%%%%%%%%%%%%%%%%%%%%%%%%%%%%%%%%%%%%%%%%%%%%%%%%%%
% Links and references
%%%%%%%%%%%%%%%%%%%%%%%%%%%%%%%%%%%%%%%%%%%%%%%%%%%%%%%
\definecolor{citec}{rgb}{0.0, 0.33, 0.71}
\definecolor{defc}{rgb}{0.0, 0.5, 0.0}

\usepackage{url}
\usepackage{varioref}
\usepackage[hyperfootnotes=false, 
pdftex,
pdfauthor	= {},
pdftitle	= {},
pdfsubject	= {},
pdfkeywords	= {}]{hyperref}
\hypersetup{
	colorlinks=true,
	linkcolor=black,
	filecolor=black,      
	urlcolor=citec,
	citecolor=citec
}
\usepackage{cleveref}


%%%%%%%%%%%%%%%%%%%%%%%%%%%%%%%%%%%%%%%%%%%%%%%%%%%%%%%
% Extra packages
%%%%%%%%%%%%%%%%%%%%%%%%%%%%%%%%%%%%%%%%%%%%%%%%%%%%%%%
\usepackage{acronym}						% To display a list of defined acronyms
\usepackage{adjustbox}						% Can be used to rezise tables and graphs
\usepackage{booktabs}						% For nicer tables
\usepackage{multirow}						% For multirow cells in tables
\usepackage{tabularx}						% For tabularx tables where X cloumns auto adjust to long content


%%%%%%%%%%%%%%%%%%%%%%%%%%%%%%%%%%%%%%%%%%%%%%%%%%%%%%%
% Define line spacing
%%%%%%%%%%%%%%%%%%%%%%%%%%%%%%%%%%%%%%%%%%%%%%%%%%%%%%%
%\renewcommand{\baselinestretch}{1.25}
\usepackage{setspace}
\setlength{\parindent}{0em}
\setlength{\parskip}{0.5em}

\usepackage{enumitem}
\setlist[1]{topsep=2pt, itemsep=3pt, parsep=4pt}
\setlist[2]{topsep=0pt, itemsep=1pt, parsep=2pt}
\setlist[3]{topsep=-1pt, noitemsep}
%\newenvironment{myitemize}{%
%	\vspace{-0.2cm}
%	\begin{itemize}
%		\setlength{\itemsep}{0pt}
%		\setlength{\parskip}{0pt}
%		\setlength{\parsep}{0pt}     }
%	{ \end{itemize}                  }


%%%%%%%%%%%%%%%%%%%%%%%%%%%%%%%%%%%%%%%%%%%%%%%%%%%%%%%
% BibLaTeX
%%%%%%%%%%%%%%%%%%%%%%%%%%%%%%%%%%%%%%%%%%%%%%%%%%%%%%%
\usepackage[style=apa, natbib=true, maxbibnames=99, backend=biber]{biblatex}
\addbibresource{references/literature.bib}
%\bibliographystyle{abbrvdin.bst}
\renewcommand\bibname{References}

% Note: The following commands need the "url" package
\setcounter{biburlnumpenalty}{100}
\setcounter{biburlucpenalty}{100}
\setcounter{biburllcpenalty}{100}


%%%%%%%%%%%%%%%%%%%%%%%%%%%%%%%%%%%%%%%%%%%%%%%%%%%%%%%
% Define the header and footer
%%%%%%%%%%%%%%%%%%%%%%%%%%%%%%%%%%%%%%%%%%%%%%%%%%%%%%%
%\pagestyle{plain}
\usepackage{lastpage}									% Allows for the total page count (last page number) to be used
\usepackage{fancyhdr}                                   % Includes the fancy header package

\let\MakeUppercase\relax
\renewcommand{\chaptermark}[1]{\markboth{#1}{}}			% Should make the \chaptermark command work
\renewcommand{\sectionmark}[1]{\markright{#1}}			% Removes the section number from the \rightmark command

\fancypagestyle{fancy}{%                                % Redifines the page style "fancy", which is used on normal pages
   \fancyhf{}                                           % Clears all header and footers
   \renewcommand{\headrulewidth}{1pt}                   % Sets the toprule
   \renewcommand{\footrulewidth}{0pt}                   % Removes the bottomrule
   \lhead{\leftmark :\ \rightmark}                      % Sets the left header to "chaper name: section name"
   \rhead{Page \thepage{}}								% Sets the right header to the page number
}

% Redefine the plain page style
\fancypagestyle{plain}{%								% Redifines the page style "plain", which is used on the first page of a new chapter
  \fancyhf{}%											% Clears all header and footers
  \renewcommand{\headrulewidth}{1pt}                    % Line at the header visible
  \renewcommand{\footrulewidth}{0pt}                    % Line at the footer invisible
  \lhead{\chaptername\ \thechapter\ $-$\ \leftmark}     % Sets the left footer to the chaper name
  \rhead{Page \thepage{}}                               % Sets the right footer to the pager number
}

% Define a new style for the content lists
\fancypagestyle{referencelist}{%
	\fancyhf{}
	\lhead{}
	\rhead{}  
	\cfoot{\thepage}
	\renewcommand{\headrulewidth}{1pt}
	\renewcommand{\footrulewidth}{0pt}
}

% Define a new style for the appendix
\fancypagestyle{appendix}{%
	\fancyhf{}
	\lhead{Appendix}
	\rhead{Page \thepage{}}
	\renewcommand{\headrulewidth}{1pt}
	\renewcommand{\footrulewidth}{0pt}
}


%%%%%%%%%%%%%%%%%%%%%%%%%%%%%%%%%%%%%%%%%%%%%%%%%%%%%%%
% Chapter Title Design
%%%%%%%%%%%%%%%%%%%%%%%%%%%%%%%%%%%%%%%%%%%%%%%%%%%%%%%
\makeatletter
   \def\@makechapterhead#1{%
     %%%%\vspace*{50\p@}% %%% removed!
     {\parindent \z@ \raggedright \normalfont
       \ifnum \c@secnumdepth >\m@ne
           \huge\bfseries \@chapapp\space \thechapter
           \par\nobreak
           \vskip 20\p@		% Skip between Chapter and Section
       \fi
       \interlinepenalty\@M
      \Huge \bfseries #1\par\nobreak
       \vskip 40\p@			% Skip between Section an text
     }}
   \def\@makeschapterhead#1{%
     %%%%%\vspace*{50\p@}% %%% removed!
     {\parindent \z@ \raggedright
       \normalfont
       \interlinepenalty\@M
       \Huge \bfseries  #1\par\nobreak
       \vskip 30\p@			% Skip between Table Of (Contents) and the table
     }}
\makeatother


%%%%%%%%%%%%%%%%%%%%%%%%%%%%%%%%%%%%%%%%%%%%%%%%%%%%%%%%%%%%%%%%%
% Titlepage
%%%%%%%%%%%%%%%%%%%%%%%%%%%%%%%%%%%%%%%%%%%%%%%%%%%%%%%%%%%%%%%%%
\newcommand{\UDOTitle}[9]{
	
	\thispagestyle{empty}
	\includegraphics[width=3in]{figures/tud_logo_rgb.jpg}
	\vspace*{\stretch{1}}
	\\
	{\parindent0cm
		\rule{\linewidth}{.1ex}}
	\begin{flushright}
		\vspace*{\stretch{1}}
		\sffamily\bfseries\Huge
		#1\\
		\vspace*{\stretch{1}}
		\sffamily\bfseries\large
		#2\\
		\vspace*{\stretch{1}}
	\end{flushright}
	\rule{\linewidth}{.1ex}
	
	\vspace*{\stretch{1}}
	\begin{center}
		
		\vspace*{\stretch{1}}
		\large #5\\
		
		\vspace*{\stretch{1}}
		\large Study Program: #4 \\[1mm]
		\large Enrollment Number: #3\\[1mm]
		\large First Reviewer:  #8 \\[1mm]
		\large Second Reviewer:  #9 \\[1mm]
		
		\vspace*{\stretch{1}}
		\large Working Time: #6
		
		\vspace*{\stretch{2}}
		\large Chair of #7\\
		\large Department of Computer Science\\
		\large TU Dortmund University\\
	\end{center}
}


%%%%%%%%%%%%%%%%%%%%%%%%%%%%%%%%%%%%%%%%%%%%%%%%%%%%%%%
% Figures
%%%%%%%%%%%%%%%%%%%%%%%%%%%%%%%%%%%%%%%%%%%%%%%%%%%%%%%
%\usepackage{float}
%\usepackage{wrapfig}

% Note: \usepackage{subcaption} not allowed here

%\usepackage{tikz}
%\usetikzlibrary{trees}

\usepackage{pdfpages}


%%%%%%%%%%%%%%%%%%%%%%%%%%%%%%%%%%%%%%%%%%%%%%%%%%%%%%%
% Code
%%%%%%%%%%%%%%%%%%%%%%%%%%%%%%%%%%%%%%%%%%%%%%%%%%%%%%%
\usepackage[plain,chapter]{algorithm}
\usepackage{algorithmic}
\makeatletter
\renewcommand{\ALG@name}{Algorithm}
\makeatother

\renewcommand{\listalgorithmname}{List of Algorithms}

%\renewcommand{\algorithmicrequire}{\textit{Eingabe:}}
%\renewcommand{\algorithmicensure}{\textit{Ausgabe:}}
%\floatname{algorithm}{Algorithmus}
%\renewcommand{\listalgorithmname}{Algorithmenverzeichnis}
%\renewcommand{\algorithmiccomment}[1]{\color{grau}{// #1}}

%\usepackage{minted}
\usepackage{listings}

\definecolor{dkgreen}{rgb}{0,0.6,0}
\definecolor{gray}{rgb}{0.5,0.5,0.5}
\definecolor{orange}{rgb}{1,0.5,0}

\lstset{frame=tb,
  language=Java,
  aboveskip=3mm,
  belowskip=3mm,
  showstringspaces=false,
  columns=flexible,
  basicstyle={\small\ttfamily},
  numbers=left,
  numberstyle=\tiny\color{gray},
  keywordstyle=\color{orange},
  keywords=[2]{PARAMETER_MEASURE_TYPES, MEASURE_TYPES, NUMERICAL_MEASURES_TYPE, PARAMETER_NUMERICAL_MEASURE},
  keywordstyle=[2]{\color{blue}},
  commentstyle=\color{gray},
  stringstyle=\color{dkgreen},
  identifierstyle=,
  breaklines=true,
  breakatwhitespace=true,
  tabsize=3
}


%%%%%%%%%%%%%%%%%%%%%%%%%%%%%%%%%%%%%%%%%%%%%%%%%%%%%%%
% Additional definitions
%%%%%%%%%%%%%%%%%%%%%%%%%%%%%%%%%%%%%%%%%%%%%%%%%%%%%%%

% Theorem-Environments
\usepackage[amsmath,thmmarks]{ntheorem}

% Caption Package
\usepackage[margin=0pt,font=small,labelfont=bf]{caption}

% Redefine the display of the table of contents
%\setcounter{secnumdepth}{5}
%\setcounter{tocdepth}{5}

% Adjust Floating Environments anpassen
\renewcommand{\topfraction}{0.9}
\renewcommand{\bottomfraction}{0.8}

% For the book format: Adds an empty page and the next one is to the rigth
\newcommand{\blankpage}{
 \clearpage{\pagestyle{empty}\cleardoublepage}
}

% Formating the page
\usepackage{microtype}
% Do not allow single lines at the end of a page
\clubpenalty = 10000
% Do not allow single lines at the beginning of a page
\widowpenalty = 10000 \displaywidowpenalty = 10000



%%%%%%%%%%%%%%%%%%%%%%%%%%%%%%%%%%%%%%%%%%%%%%%%%%%%%%%
% Definitions for abbreviations
%%%%%%%%%%%%%%%%%%%%%%%%%%%%%%%%%%%%%%%%%%%%%%%%%%%%%%%
\usepackage{xspace}
\newcommand{\ao}{a.\nolinebreak[4]\hspace{0.125em}\nolinebreak[4]o.\@\xspace} % a.o. = among others -> unter anderem / u.a.
\newcommand{\eg}{e.\nolinebreak[4]\hspace{0.125em}\nolinebreak[4]g.\@\xspace} % e.g. = exempli gratia (latin) -> zum Beispiel / z.B.
\newcommand{\ie}{i.\nolinebreak[4]\hspace{0.125em}\nolinebreak[4]e.\@\xspace} % i.e. = id est (latin) -> das heißt / d.h.
\newcommand{\iow}{i.\nolinebreak[4]\hspace{0.125em}\nolinebreak[4]o.\nolinebreak[4]\hspace{0.125em}\nolinebreak[4]w.\@\xspace}	% iow. = in other words
\newcommand{\wrt}{w.\nolinebreak[4]\hspace{0.125em}\nolinebreak[4]r.\nolinebreak[4]\hspace{0.125em}\nolinebreak[4]t.\@\xspace} % w.r.t. = with regards to
\newcommand{\cf}{cf.\@\xspace}		% cf.  = confer (latin) -> vergleiche / vgl.


%%%%%%%%%%%%%%%%%%%%%%%%%%%%%%%%%%%%%%%%%%%%%%%%%%%%%%%
% Additional definitions
%%%%%%%%%%%%%%%%%%%%%%%%%%%%%%%%%%%%%%%%%%%%%%%%%%%%%%%

% Theorem-Environments
\usepackage[amsmath,thmmarks]{ntheorem}

% Caption Package
\usepackage[margin=0pt,font=small,labelfont=bf]{caption}

% Redefine the display of the table of contents
%\setcounter{secnumdepth}{5}
%\setcounter{tocdepth}{5}

% Adjust Floating Environments anpassen
\renewcommand{\topfraction}{0.9}
\renewcommand{\bottomfraction}{0.8}

% For the book format: Adds an empty page and the next one is to the rigth
%\newcommand{\blankpage}{
%	\clearpage{\pagestyle{empty}\cleardoublepage}
%}

% Formating the page
\usepackage{microtype}
% Do not allow single lines at the end of a page
\clubpenalty = 10000
% Do not allow single lines at the beginning of a page
\widowpenalty = 10000 \displaywidowpenalty = 10000


%%%%%%%%%%%%%%%%%%%%%%%%%%%%%%%%%%%%%%%%%%%%%%%%%%%%%%%
% Umgebungen
%%%%%%%%%%%%%%%%%%%%%%%%%%%%%%%%%%%%%%%%%%%%%%%%%%%%%%%
\usepackage[framemethod=TikZ]{mdframed}

% Style %
\mdfdefinestyle{enviStyle}{
	innertopmargin = 10pt,
	linewidth      = 1pt,
	frametitlerule = true,
	roundcorner    = 2pt%
}

% Counter für Definitionen und Beispiele %
\newcounter{def}[section]\setcounter{def}{0}
\renewcommand{\thedef}{\arabic{section}.\arabic{def}}
\newcounter{bsp}[section]\setcounter{bsp}{0}
\renewcommand{\thebsp}{\arabic{section}.\arabic{bsp}}
\newcounter{pro}[section]\setcounter{pro}{0}
\renewcommand{\thepro}{\arabic{section}.\arabic{pro}}

\crefname{def}{definition}{definitions}
\Crefname{def}{Definition}{Definition}
\crefname{bsp}{example}{examples}
\Crefname{bsp}{Example}{Examples}
\crefname{pro}{proposition}{propositions}
\Crefname{pro}{Proposition}{Propositions}

% Environment mit counter mit zB \newenvironment{name}[args]{begin_def}{end_def}
\newenvironment{CountingProposition}[2][]{%
	\refstepcounter{pro}%
	\ifstrempty{#1}%
	{\mdfsetup{%
			frametitle={{\strut Proposition~\thepro}}}
	}%
	{\mdfsetup{%
			frametitle={{\strut Proposition~\thepro:~#1}}}%
	}%
	\mdfsetup{
		nobreak                   = true,
		linecolor                 = gray,
		frametitlebackgroundcolor = gray!50,
		style                     = enviStyle
	}
	\begin{mdframed}[]\relax%
		\label{#2}}{\end{mdframed}}

\newenvironment{CDef}[2][]{%
	\refstepcounter{def}%
	\ifstrempty{#1}%
	{\mdfsetup{%
			frametitle={{\strut Definition~\thedef}}}
	}%
	{\mdfsetup{%
			frametitle={{\strut Definition~\thedef:~#1}}}%
	}%
	\mdfsetup{
		nobreak                   = false,
		linecolor                 = defc,
		frametitlebackgroundcolor = defc!50,
		style                     = enviStyle
	}
	\begin{mdframed}[]\relax%
		\label{#2}}{\end{mdframed}}

\newenvironment{CDefNB}[2][]{%
	\refstepcounter{def}%
	\ifstrempty{#1}%
	{\mdfsetup{%
			frametitle={{\strut Definition~\thedef}}}
	}%
	{\mdfsetup{%
			frametitle={{\strut Definition~\thedef:~#1}}}%
	}%
	\mdfsetup{
		nobreak                   = true,
		linecolor                 = defc,
		frametitlebackgroundcolor = defc!50,
		style                     = enviStyle
	}
	\begin{mdframed}[]\relax%
		\label{#2}}{\end{mdframed}}

\newenvironment{CExample}[2][]{%
	\refstepcounter{bsp}%
	\ifstrempty{#1}%
	{\mdfsetup{%
			frametitle={{\strut Example~\thebsp}}}
	}%
	{\mdfsetup{%
			frametitle={{\strut Example~\thebsp:~#1}}}%
	}%
	\mdfsetup{
		nobreak                   = false,
		linecolor                 = citec!50,
		frametitlebackgroundcolor = citec!20,
		style                     = enviStyle
	}
	\begin{mdframed}[]\relax%
		\label{#2}}{\end{mdframed}}

\newenvironment{CExampleNB}[2][]{%
	\refstepcounter{bsp}%
	\ifstrempty{#1}%
	{\mdfsetup{%
			frametitle={{\strut Example~\thebsp}}}
	}%
	{\mdfsetup{%
			frametitle={{\strut Example~\thebsp:~#1}}}%
	}%
	\mdfsetup{
		nobreak                   = true,
		linecolor                 = citec!50,
		frametitlebackgroundcolor = citec!20,
		style                     = enviStyle
	}
	\begin{mdframed}[]\relax%
		\label{#2}}{\end{mdframed}}