% !TEX root = ../main.tex
\chapter{Example Chapter} \label{chapter:example_chapter}

Please note that there are different ways to cite your sources:
\begin{enumerate}
	\item Within your text, e.g. ``\citet[81]{hammer1993reengineering} showed that $\dots$"
	\item As internet source: \citet{diewi}
	\item Seperated: 
		\begin{itemize}
			\item With \texttt{cite} (page no. are only possible with a single source):
				\begin{itemize}
					\item \cite[685]{janiesch2021machine}
					\item \cite{janiesch2021machine, herm2021symbolic}
				\end{itemize}
			\item With \texttt{cites}: 
				\begin{itemize}
					\item \cites[685]{janiesch2021machine}
					\item \cites[685]{janiesch2021machine}[289]{herm2021symbolic}
				\end{itemize}
			\item With \texttt{citep} (page no. are only possible with a single source): 
				\begin{itemize}
					\item \citep[685]{janiesch2021machine}
					\item \citep{janiesch2021machine, herm2021symbolic}
				\end{itemize}
		\end{itemize}
	\item And references within this document are done by setting labels and using them as:
		\begin{itemize}
			\item ``\autoref{chapter:introduction}" to auto fill in the thing (chapter /  section / figure / etc.) the label is referencing
			\item ``\cref{chapter:introduction}" does the same by using the \texttt{cleverref} package, which supports more labels and things like ``\crefrange*{chapter:introduction}{chapter:conclusion}"
			\item ``chapter~\ref{chapter:introduction}" where the referenced thing is manually written in front of its number
		\end{itemize}
\end{enumerate}

%\lipsum[14-14]



\section{Section A}
\label{subsection:a}

\lipsum[15-18]



\section{Section B}
\label{subsection:b}

\subsection{Subsection B.a}
\lipsum[19-21]

\subsection{Subsection B.b}
\lipsum[22-23]